\chapter{Current APRS Demodulation Approaches}

\section{Terminal Node Controllers}
%Currently copied from old hardware based demodulators
Currently there are many systems that will demodulate these Bell 202 encoded APRS packets. The original hardware used for this communication style was dedicated modems similar to dial-up 56k modems that did the encoding and decoding. These modems were connected directly to the radio and the radio would let the modem know through a signal pin if it was receiving what it thought was a data carrying signal. The terminal node controller (TNC), a modem used in AX.25 operation, would let the radio know by signaling the radio to transmit when it had data to send out. As mentioned in the introduction and early in this section, the technologies that are being used for this data transport originated in the late 1980s even though the spec for the protocol itself did not get officially released until 2000. This serves as a reminder that old hardware is commonly used that is not well understood. If a user gets something reasonable working, they will use it without doing a full analysis on edge cases or making simple modification to improve performance.

With a radio and a Terminal Node Controller, digipeaters (digital packet repeaters) are possible. Within the testing scope of this project there are two TNCs that used in order to be able to compare our results to them which were the AEA PK-88 and the Kantronics KAM Plus. Digipeaters are an essential part of the ham packet network, but many users wish to report their GPS position onto the APRS network instead of just relaying traffic for other stations. In order to accomplish this, a GPS receiver is required. Now, stations can take the data from their GPS receiver and put it in the payload of the APRS packet and transmit the GPS reported position onto the network. However, the PK-88 and the KAM Plus although used very frequently in APRS systems are not fully dedicated hardware for APRS, but instead modems that are being used for ARPS.

\section{Specialized APRS Hardware}
%Currently copied from Dedicated Hardware
Many people know exactly what they would like to do with APRS and exactly what traffic they want to contribute to the APRS network. This has initiated dedicated hardware for APRS with a UI in order to make it simple for the end users to quickly run and configure their APRS stations. Some examples of APRS exclusive devices are ArgentData�s OpenTrackers, Byonics� TinyTrack, and Fox Delta�s Fox Track. These compact packages along with a radio and a GPS module perform APRS tasks at a satisfactory level for many users.

Since the average user only wants to report positional information, these dedicated devices make it simple to setup. These trackers contain many features, but they don�t implement the full APRS specification. An example is the messaging service since these devices don�t have a display for the message. Certain radio manufacturers have begun to integrating the TNCs into the radios themselves to utilize the radio�s screen. The Kenwood TM-D700 series and Yaesu FTM-350 are examples.

However, both the options that were presented in section 2.1 and 2.2 require going out and buying special hardware in order to perform APRS. This can be expensive and cost prohibitive for some hams to be able to begin APRS operations.

\section{Software Based Demodualtion}
%Copied from old Software Based Demodulation
It can be assumed that before a ham operator becomes interested in the APRS network and sending APRS packets that they will already have a radio. So, if they already have a radio all they have to do is buy a piece of hardware that will do the modulation in order to send a packet. However, hardware costs money and as hams are at least somewhat technology savvy most have computers. A software solution seeks to take advantage of the computer already in the operator�s possession to do the modulation and demodulation and hence offer a cheaper alternative.

This seems to be a route that some are taking and a demodulation scheme that this project explores in detail, but first some more information on current systems that operate in this software realm. Some examples of the software that can be used are George Rossopoylos�s Packet Engine or Thomas Sailer�s Linux Sound Modem. On a computer, even ones with minimal resources, there are algorithms that are being used to demodulate the APRS packets. Again, what this project aims to investigate is if improvements can be made to the algorithms used to decode these packets in order to make the software based systems more robust and as good as dedicated hardware. Preliminary testing shows that the software still has room for improvement in order to be at least as good as dedicated hardware.

%!!!!!!!!!!!!NEED MOTIVATION... EXPLAING PROS OF SOFTWARE HERE...!!!!!!!!!!!!!
%As mentioned thus far, software provides a viable low cost alternative to dedicated hardware, but is just that - a cheap solution. It still has some weaknesses that need to be addressed, and refinements to be made to improve demodulation to match performance of hardware. 