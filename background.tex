\chapter{Background}

APRS is a local area awareness network designed from hams (amateur radio operators), by hams. There is a plethora of features supported within the specification of APRS including the sending of messages, bulletins, meteorological data, waypoints, objects, and most commonly GPS locations. Many of the capabilities have not been fully implemented and other implementation details are inconsistent. The goal of this research is not to discuss the shortcoming in the APRS protocol itself which lies within its implementation and use on top of the AX.25 protocol. Instead this research focuses on Layer 1 problems in the Open System Interconnection (OSI) model. 
The Layer 1 protocol for ARPS is based off of a Bell 202 modem. The Bell 202 modem was patented in 1984 using 1200Hz and 2200Hz tones, although the patent was originally filed in 1981 [1]. Interestingly, the international telecommunication union didn’t publish a standard for these modems that were used in telephone networks until 1988. In the standard, however they use 1300Hz and 2100Hz tones for symbol 1 (called a mark) and symbol 0 (called a space) respectively [2]. The basic modulation scheme is that the marks and spaces are used in a non-return to zero inverted (NRZI) encoding for the actual data. When a transition occurs from one symbol to another that symbolizes a ‘0’ bit in the original data bit stream and if the symbol remains constant over multiple symbol periods, that signifies a ‘1’ bit in the original data bit stream. It is also worth noting that there will be no more than 5 consecutive symbols during a packet, as the scheme will bit stuff in order to make sure that synchronization on the clock can be maintained. The only exception to this rule is for the flag(s) that mark the start and end of a packet. The AX.25 flag is hex 0x7E without bit stuffing. Common practice is to send multiple flags consecutively to give transmitting radio time to key up and settle and to give receiving radios time for their squelch to open.
This encoding and modulation is used to transmit the data of the APRS packets, but further background is needed in the APRS environment to discuss where this research applies. As mentioned this will not be looking into the APRS protocol itself but more focused on the Bell 202 modem and specifically the demodulation aspects. Implementing a Bell 202 modulator is simple in comparison to making a robust demodulator - especially when trying to implement one in software.

\section{Hardware Based Demodulators}

Currently there are many systems that will demodulate these Bell 202 encoded APRS packets. The original hardware used for this communication style was dedicated modems similar to dial-up 56k modems that did the encoding and decoding. These modems were connected directly to the radio and the radio would let the modem know through a signal pin if it was receiving what it thought was a data carrying signal. The terminal node controller (TNC), a modem used in AX.25 operation, would let the radio know by signaling the radio to transmit when it had data to send out. As mentioned in the introduction and early in this section, the technologies that are being used for this data transport originated in the late 1980s even though the spec for the protocol itself did not get officially released until 2000. This serves as a reminder that old hardware is commonly used that is not well understood. If a user gets something reasonable working, they will use it without doing a full analysis on edge cases or making simple modification to improve performance.
With a radio and a Terminal Node Controller, digipeaters (digital packet repeaters) are possible. Within the testing scope of this project there are two TNCs that used in order to be able to compare our results to them which were the AEA PK-88 and the Kantronics KAM Plus. Digipeaters are an essential part of the ham packet network, but many users wish to report their GPS position onto the APRS network instead of just relaying traffic for other stations. In order to accomplish this, a GPS receiver is required. Now, stations can take the data from their GPS receiver and put it in the payload of the APRS packet and transmit the GPS reported position onto the network. However, the PK-88 and the KAM Plus although used very frequently in APRS systems are not fully dedicated hardware for APRS, but instead modems that are being used for ARPS.

\section{Dedicated Hardware}

Many people know exactly what they would like to do with APRS and exactly what traffic they want to contribute to the APRS network. This has initiated dedicated hardware for APRS with a UI in order to make it simple for the end users to quickly run and configure their APRS stations. Some examples of APRS exclusive devices are ArgentData’s OpenTrackers, Byonics’ TinyTrack, and Fox Delta’s Fox Track. These compact packages along with a radio and a GPS module perform APRS tasks at a satisfactory level for many users.
Since the average user only wants to report positional information, these dedicated devices make it simple to setup. These trackers contain many features, but they don’t implement the full APRS specification. An example is the messaging service since these devices don’t have a display for the message. Certain radio manufacturers have begun to integrating the TNCs into the radios themselves to utilize the radio’s screen. The Kenwood TM-D700 series and Yaesu FTM-350 are examples.
However, both the options that were presented in section 2.1 and 2.2 require going out and buying special hardware in order to perform APRS. This can be expensive and cost prohibitive for some hams to be able to begin APRS operations.

\section{Software Based Demodulators}

It can be assumed that before a ham operator becomes interested in the APRS network and sending APRS packets that they will already have a radio. So, if they already have a radio all they have to do is buy a piece of hardware that will do the modulation in order to send a packet. However, hardware costs money and as hams are at least somewhat technology savvy most have computers. A software solution seeks to take advantage of the computer already in the operator’s possession to do the modulation and demodulation and hence offer a cheaper alternative.
This seems to be a route that some are taking and a demodulation scheme that this project explores in detail, but first some more information on current systems that operate in this software realm. Some examples of the software that can be used are George Rossopoylos’s Packet Engine or Thomas Sailer’s Linux Sound Modem. On a computer, even ones with minimal resources, there are algorithms that are being used to demodulate the APRS packets. Again, what this project aims to investigate is if improvements can be made to the algorithms used to decode these packets in order to make the software based systems more robust and as good as dedicated hardware. Preliminary testing shows that the software still has room for improvement in order to be at least as good as dedicated hardware.
