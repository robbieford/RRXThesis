\chapter{Testing Framework}

The testing of the various demodulation techniques was done against 5 different audio files. All the files were at 48000 sample rates with 16 bits per sample PCM wav encoding. 48000 Hz sample rate was selected due to being ideal for the number of samples per baud with the 1200 baud rate (40 samples). See Table 1 to see the source of each audio file. The �generated200packets.wav� file was generated modifying the JavaAX25 software to create a list of files with an incrementing counter in the payload ([KK6BQJ>APRS,WIDE1-1,WIDE2-1:!3518.11N/12039.80EWZ CNT: 10]). This is the cleanest file with the strongest signal and no emphasis and contains a total of 200 packets. The second file (�ot3Test48000Mono.wav�) was created by putting the Open Tracker 3 TNC into calibration mode, which continuously transmits a packet with a counter in the payload. The audio output of the OT3 was recorded using the microphone input of a PC. The recording captured 40 packets from this OT3 calibration mode. This file was then intentionally corrupted with ramping steps of uniform noise of 0.1 magnitude (�ot3TestwNoise48000MonoTrue.wav�) ([NOCALL>CQ:00165The quick brown fox jumps over the lazy dog]). See Figure 9.  The last two audio files are a standard testing file used by amateurs to test against. The first (�01Track1_48000.wav�) was recorded in Los Angeles over five years ago. The original file was a sample rate of 44100 and was resampled to 48000 Hz. The file is 25 minutes long with at least 1007 packets present ([WA6YLB-4>APRS,N6EX-5:$ULTW00000000----0000----000086A00001----0000000000000000\x0d\x0a]). It is unsure exactly how many packets can be properly demodulated from the recording since it is real traffic with the definitely possibility of doubling (one station talking over another). The current estimate of how many packets are in the file was calculated by counting how many potential packets can be heard in a one minute segment and then using that to extrapolate how many may be in the whole recording. Over the course of the first minute of the recording approximately 68 packets were heard which means that there is a potential total of 1750 packets in the whole recording. An example of the time domain waveform of the audio file can be seen in Figure 10. The second version of this file (�02Track2_48000.wav�) was deemphasized in a software package by WA8LMF [6]. Comparison to the software demodulation techniques were made against actual hardware devices. The audio files were played out of a PC soundcard and into devices while put into monitor mode to decode the packets. The hardware devices tested were Argent Data System�s Open Tracker 2, Open Tracker 3, Open Tracker USB, Kantronics� KAM Plus, and AEA�s PK-88. The numbers of packets decoded by each device were counted after passing through a console logger.