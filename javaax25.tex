\chapter{javaAX25}

Another software based demodulator that is available is Sivan Toledo�s JavaAX25 software package. The advantage of using this package for benchmarking different algorithms is that due to his code structure and package hierarchy it makes it simple to change different demodulation algorithms. The software is hosted on github making it convenient to access the repository. The next few paragraphs will give an overview of the features that Toledo�s software package has available in it \cite{Toledo2012,javax25github}.

JavaAX25 is a comprehensive package for doing software based modulation and demodulation of APRS packets. It includes packages for interfacing with radios, sound cards, and other standard packet programs that are used on computers; one such example is that there is a plugin to be able to use JavaAX25 with AGW Packet Engine. In addition to having all of the items that are needed to be able to do the AX.25 modulation, demodulation, and interfacing with hardware, there is also a testing framework.

From within the testing framework each aspect of the software suite can be tested. The three portions that were used most extensively in this research were the modulator, demodulator, and testing framework. In order to have a demodulator one needs to have it be a child of an abstract class that implements methods for adding individual samples to the algorithms for processing, checking to see if the current signal might be a data carrier, etc. Due to the fact that the data flow was very clear for the demodulators it made it simple for different demodulators to be implemented and then tested using the �Test� class.

The main method that is called by the class using the demodulator in order to pass the data to the algorithm is the addSamplePrivate method. This method is a way for the calling class to give data to the algorithm to be processed. Each sample is a value that corresponds to the magnitude of the audio signal at that instant in time. The samples themselves derive from the fact that digital audio is sampled at a given sample rate. For this research sample rates of both 44100 and 48000 were used, but 48000 samples per second was standardized on since it divides evenly into 40 samples per baud on a 1200bps digital encoding. The goal of demodulators it to determine when a symbol transition has occurred, once this has been determined, the time elapsed since the previous transition is used to determine how many symbols have occurred. Since consecutive symbols represent a 1 bit, the number of symbols minus one will be the number of 1 bits to add to the packet followed by a zero.

The algorithm that Toledo is currently using for the demodulation is correlation based. This is done by correlating the input signal with both a 1200Hz and a 2200Hz sine wave and seeing which of the two the input signal correlates with more. Once the correlation with each is established he filters the correlation data in order to smooth the results and make it easier to be able to pull the correct frequency out of the calculations. However, before doing any of the correlation calculations the data is passed through a band pass filter centered about 1700Hz which can be seen in Figure 2.

%This timeline helps to reinforce the fact that the technologies used for this system are old and far from cutting edge. Recently, in 2012, Sivan Toledo put together a Java based software package for demodulating these Bell 202, specifically APRS, packets. Using this package as the framework for testing other demodulation algorithms more insight was gained into software demodulation. Software demodulation provides a low cost alternative to hardware demodulators since users can run the software on hardware that they already possess. For example APRSDroid (an application in the Google�s Android Play Store) directly imports JavaAX25�s software demodulation \cite{APRSdroid}. 