%Import the Thesis formatting
\documentclass[12pt]{ucthesis}

%Package Imports
\usepackage[pdftex]{graphicx}
\usepackage{parskip}


\begin{document}
\title{Improvements and Analysis of Software Based 1200 Baud Audio Frequency Shift Keying Demodulation}
\author{Robert Campbell}
\degreemonth{June} \degreeyear{2014} \degree{Master of Science}
\defensemonth{June} \defenseyear{2014}
\numberofmembers{3} \chair{John Bellardo, Ph.D.} \othermemberA{Bridget Benson, Ph.D.} \othermemberB{Dennis Derickson, Ph.D.} \field{Computer Science} \campus{San Luis Obispo}
\copyrightyears{seven}
\maketitle

\begin{frontmatter}

\copyrightpage

\committeemembershippage

\begin{abstract}
Digital communications continues to be an interesting field of study as new technologies appear and old methodologies get revisited or renovated. The goal of this research was to look into the old digital communication scheme of Bell 202 and find a way to demodulate the signal in software with results similar to those of hardware demodulators. The research shows that through using Sivan Toledo’s JavaAX25 software package, new demodulation algorithms can be implemented that decode Bell 202 AX.25 encoded packets that the existing software tool could not.
\end{abstract}


\tableofcontents

\end{frontmatter}

\chapter{Introduction}

Amateur Radio Operators, commonly referred to as "hams," make the best of what resources they have available to them. However, once something is working a "don't touch it if it ain't broke" approach is often taken. Between these two mentalities some interesting phenomenon have occurred within the ham community. For example, some radio systems that are in active use today have only seen very minimal attention since the 1980's when they were originally installed. On the flip side of leaving things alone, hams are very quick to take advantage of a good deal or opportunity when they are presented one.The implementation and development of the Automated Packet Reporting System (APRS) is no exception to the way hams approach things \cite{Bruninga}. Much of this system is based off older hardware and protocols - from the 1980s - that was readily available and few improvements have been made, and although the specification has been relatively stable there are inconsistencies. These inconsistencies include varying implementations from vendor to vendor as well as portions of the specification that are not clearly defined resulting in vastly inconsistent performance \cite{KWFThesis, KWFTAPR}.

%A recent scenario that expresses both of these characteristics of hams is when the Federal Communications Commission (FCC) required that public service agencies such as police, fire, and ambulance go to narrow banding (a 12.5kHz channel from a 25kHz channel). This change was needed to be able to support more channels in the limited radio frequency spectrum \cite{Commission2012}. Due to this change a lot of equipment became available that could not do narrow banding, which many hams sprung at the opportunity for. In addition to showing their need to take advantage of a good opportunity, it also shows that they do not feel the need to upgrade their systems even though there are congested areas that would benefit immensely from narrow banding.

So, what is APRS, and why does it matter? A brief introduction to APRS is that it is a digital communication scheme used by hams where a packet (whose content is varied, but is usually a GPS position - which is what gave APRS it's original name "Automated Position Reporting System"\cite{WikiAPRS}) is sent out over radio. A major challenge to this protocol and method of digital communication is the fact that it uses radio, which is susceptible to interference, weak signals as the distance from the transmitting station increases, as well as a myriad of other items. This research focuses specifically on the receiving end of these signals in order to see what improvements can be made to software based approaches to decoding - demodulating - these packets.

The reasoning for trying to make improvements in software based demodulation are many, but a few of the more motivational ones are to follow. One advantage of doing software based demodulation is that it removes the necessity of specialty hardware; Instead of having dedicated hardware whose sole purpose is to modulate and demodulate APRS packets, hams can use a computer to do these tasks. By using a computer's sound card, audio from the radio can be processed using software to decode received packets, or audio can be played from the sound card to the radio to be transmitted. With the abundance of personal computers, this can provide a much cheaper solution for hams who are interested in trying out APRS without having to put down a potentially big initial investment (\textasciitilde\$200 \cite{Kantronics2014,Outlet2014}) for a piece of hardware that serves one purpose. The price of this specialty hardware is steep and it is limited to only performing communication on a single channel. When using a line in / out on a computer they are typically stereo meaning that a single sound card could handle operations on multiple channels. If two channels just is not enough the capabilities of a computer demodulator can be expanded my merely adding another sound card which is relatively cheap at \textasciitilde\$20 \cite{Newegg}. To perform communication on 4 channels using dedicated hardware the cost would be $800! For this cost a whole computer with a half dozen sound cards could be purchased, only further expanding capabilities.

%Another cost advantage is when multiple APRS networks are operated alongside each other. For some events that hams assist with, GPS tracking is used to keep track of assets, such as support vehicles, ambulances, water trucks, etc. In order to get more frequent position updates the traffic can be moved from the primary transmit frequency to a different back haul frequency to alleviate some of the Radio Frequency (RF) congestion. If there are multiple back haul frequencies - three, for example - in addition to the primary transmitting frequency, there would be a total of four frequencies carrying APRS traffic that need to be monitored and the traffic demodulated. Instead of spending \$800 to get four dedicated units for APRS to handle all of the traffic an extra sound card could be purchased for a computer (\textasciitilde\$20 \cite{Newegg}). Since audio inputs are typically stereo with a left and a right channel, these can be considered as two separate audio inputs since the radio transmissions are mono (single channel). Hence, between the input that the computer already has and the new one added through the extra sound card, there could be a total of four audio input channels on the computer which the audio from the radios can be piped into.

In addition the the price advantages of software based demodulation approaches there is also one other primary advantage. If software is being used instead of hardware there is the potential for a lot more capabilities since processing power and available memory increase drastically. For instance, on one of the dedicated hardware solutions, the Kantronics KPC-3 Plus, has a mere 512KB of memory compared to that of any computer which is over 4GB as of 2014 - and that is just the ram, not the hard drive space \cite{Kantronics2014,Graham-Smith2014}. Additionally, instead of just being able to handle live events and process each data point in the best manner possible as soon as it comes in, post processing becomes an option.

With the cost and versatility of a software demodulation solution now introduced, the paper addresses the following: Chapter 2 goes into background information, with a deeper introduction to APRS and a presentation of the aspects important to understanding this research. In Chapter 3, some of the current methods for interfacing with APRS, both hardware and software, are explained. Demodulation techniques are discussed in Chapter 4. Chapter 5 talks about the challenges of demodulating APRS packets. Chapter 6 discusses the methods used for benchmarking and comparing the demodulators. In Chapter 7, information on how the demodulators and algorithms are tested is presented. Chapter 8 goes into more detail about the implementations in this project. Chapter 9 discusses the results of both the newly implemented algorithms and compares them to other demodulators. Areas of additional research and future work are discussed in Chapter 10. Chapter 11 is concluding remarks.

\chapter{Background}

APRS is a local area awareness network designed from hams (amateur radio operators), by hams. There is a plethora of features supported within the specification of APRS including the sending of messages, bulletins, meteorological data, waypoints, objects, and most commonly GPS locations. Many of the capabilities have not been fully implemented and other implementation details are inconsistent. The goal of this research is not to discuss the shortcoming in the APRS protocol itself which lies within its implementation and use on top of the AX.25 protocol. Instead this research focuses on Layer 1 problems in the Open System Interconnection (OSI) model.

The Layer 1 protocol for ARPS is based off of a Bell 202 modem. The Bell 202 modem was patented in 1984 using 1200Hz and 2200Hz tones, although the patent was originally filed in 1981 [1]. Interestingly, the international telecommunication union didn’t publish a standard for these modems that were used in telephone networks until 1988. In the standard, however they use 1300Hz and 2100Hz tones for symbol 1 (called a mark) and symbol 0 (called a space) respectively [2]. The basic modulation scheme is that the marks and spaces are used in a non-return to zero inverted (NRZI) encoding for the actual data. When a transition occurs from one symbol to another that symbolizes a ‘0’ bit in the original data bit stream and if the symbol remains constant over multiple symbol periods, that signifies a ‘1’ bit in the original data bit stream. It is also worth noting that there will be no more than 5 consecutive symbols during a packet, as the scheme will bit stuff in order to make sure that synchronization on the clock can be maintained. The only exception to this rule is for the flag(s) that mark the start and end of a packet. The AX.25 flag is hex 0x7E without bit stuffing. Common practice is to send multiple flags consecutively to give transmitting radio time to key up and settle and to give receiving radios time for their squelch to open.

This encoding and modulation is used to transmit the data of the APRS packets, but further background is needed in the APRS environment to discuss where this research applies. As mentioned this will not be looking into the APRS protocol itself but more focused on the Bell 202 modem and specifically the demodulation aspects. Implementing a Bell 202 modulator is simple in comparison to making a robust demodulator - especially when trying to implement one in software.

\section{Hardware Based Demodulators}

Currently there are many systems that will demodulate these Bell 202 encoded APRS packets. The original hardware used for this communication style was dedicated modems similar to dial-up 56k modems that did the encoding and decoding. These modems were connected directly to the radio and the radio would let the modem know through a signal pin if it was receiving what it thought was a data carrying signal. The terminal node controller (TNC), a modem used in AX.25 operation, would let the radio know by signaling the radio to transmit when it had data to send out. As mentioned in the introduction and early in this section, the technologies that are being used for this data transport originated in the late 1980s even though the spec for the protocol itself did not get officially released until 2000. This serves as a reminder that old hardware is commonly used that is not well understood. If a user gets something reasonable working, they will use it without doing a full analysis on edge cases or making simple modification to improve performance.

With a radio and a Terminal Node Controller, digipeaters (digital packet repeaters) are possible. Within the testing scope of this project there are two TNCs that used in order to be able to compare our results to them which were the AEA PK-88 and the Kantronics KAM Plus. Digipeaters are an essential part of the ham packet network, but many users wish to report their GPS position onto the APRS network instead of just relaying traffic for other stations. In order to accomplish this, a GPS receiver is required. Now, stations can take the data from their GPS receiver and put it in the payload of the APRS packet and transmit the GPS reported position onto the network. However, the PK-88 and the KAM Plus although used very frequently in APRS systems are not fully dedicated hardware for APRS, but instead modems that are being used for ARPS.

\section{Dedicated Hardware}

Many people know exactly what they would like to do with APRS and exactly what traffic they want to contribute to the APRS network. This has initiated dedicated hardware for APRS with a UI in order to make it simple for the end users to quickly run and configure their APRS stations. Some examples of APRS exclusive devices are ArgentData’s OpenTrackers, Byonics’ TinyTrack, and Fox Delta’s Fox Track. These compact packages along with a radio and a GPS module perform APRS tasks at a satisfactory level for many users.

Since the average user only wants to report positional information, these dedicated devices make it simple to setup. These trackers contain many features, but they don’t implement the full APRS specification. An example is the messaging service since these devices don’t have a display for the message. Certain radio manufacturers have begun to integrating the TNCs into the radios themselves to utilize the radio’s screen. The Kenwood TM-D700 series and Yaesu FTM-350 are examples.

However, both the options that were presented in section 2.1 and 2.2 require going out and buying special hardware in order to perform APRS. This can be expensive and cost prohibitive for some hams to be able to begin APRS operations.

\section{Software Based Demodulators}

It can be assumed that before a ham operator becomes interested in the APRS network and sending APRS packets that they will already have a radio. So, if they already have a radio all they have to do is buy a piece of hardware that will do the modulation in order to send a packet. However, hardware costs money and as hams are at least somewhat technology savvy most have computers. A software solution seeks to take advantage of the computer already in the operator’s possession to do the modulation and demodulation and hence offer a cheaper alternative.

This seems to be a route that some are taking and a demodulation scheme that this project explores in detail, but first some more information on current systems that operate in this software realm. Some examples of the software that can be used are George Rossopoylos’s Packet Engine or Thomas Sailer’s Linux Sound Modem. On a computer, even ones with minimal resources, there are algorithms that are being used to demodulate the APRS packets. Again, what this project aims to investigate is if improvements can be made to the algorithms used to decode these packets in order to make the software based systems more robust and as good as dedicated hardware. Preliminary testing shows that the software still has room for improvement in order to be at least as good as dedicated hardware.

\chapter{javaAX25}

Another software based demodulator that is available is Sivan Toledo�s JavaAX25 software package. The advantage of using this package for benchmarking different algorithms is that due to his code structure and package hierarchy it makes it simple to change different demodulation algorithms. The software is hosted on github making it convenient to access the repository. The next few paragraphs will give an overview of the features that Toledo�s software package has available in it (((REFERENCES 4, 5))).

JavaAX25 is a comprehensive package for doing software based modulation and demodulation of APRS packets. It includes packages for interfacing with radios, sound cards, and other standard packet programs that are used on computers; one such example is that there is a plugin to be able to use JavaAX25 with AGW Packet Engine. In addition to having all of the items that are needed to be able to do the AX.25 modulation, demodulation, and interfacing with hardware, there is also a testing framework.

From within the testing framework each aspect of the software suite can be tested. The three portions that were used most extensively in this research were the modulator, demodulator, and testing framework. In order to have a demodulator one needs to have it be a child of an abstract class that implements methods for adding individual samples to the algorithms for processing, checking to see if the current signal might be a data carrier, etc. Due to the fact that the data flow was very clear for the demodulators it made it simple for different demodulators to be implemented and then tested using the �Test� class.

The main method that is called by the class using the demodulator in order to pass the data to the algorithm is the addSamplePrivate method. This method is a way for the calling class to give data to the algorithm to be processed. Each sample is a value that corresponds to the magnitude of the audio signal at that instant in time. The samples themselves derive from the fact that digital audio is sampled at a given sample rate. For this research sample rates of both 44100 and 48000 were used, but 48000 samples per second was standardized on since it divides evenly into 40 samples per baud on a 1200bps digital encoding. The goal of demodulators it to determine when a symbol transition has occurred, once this has been determined, the time elapsed since the previous transition is used to determine how many symbols have occurred. Since consecutive symbols represent a 1 bit, the number of symbols minus one will be the number of 1 bits to add to the packet followed by a zero.

The algorithm that Toledo is currently using for the demodulation is correlation based. This is done by correlating the input signal with both a 1200Hz and a 2200Hz sine wave and seeing which of the two the input signal correlates with more. Once the correlation with each is established he filters the correlation data in order to smooth the results and make it easier to be able to pull the correct frequency out of the calculations. However, before doing any of the correlation calculations the data is passed through a band pass filter centered about 1700Hz which can be seen in Figure 2.

\chapter{Implementation}

This chapter will go through all of the implementation details of each algorithm implemented. They will be presented in order of simplicity, with the more intricate ones presented last. This also will introduce them mostly chronologically since a naive approaches allowed for more insight to be gained into the JavaAX25 software package before implementing more complicated algorithms. In addition to giving a brief overview of each implementation some performance data will be provided, but all of the data will be presented in the Results Chapter. To see detailed 

\section{Strict Zero Crossing Demodulator}
This approach used the technique of finding zero crossings and then using those to determine the period. From the period the frequency was then calculated. For 1200Hz and 2200Hz tones zero crossings are expected every 833�s and 455�s respectively. If it was above 1700Hz it was assumed that a mark was present in the signal and if lower than 1700Hz a space must be present. The zero crossing were found by determined if the signal was negative and changed to positive or if it was positive and changed to negative. Although this algorithm was only able to decode a little over half of the packets as some of the other algorithms, it proved to be an important stepping stone into javAX25 and allowed for preparation into restructuring the project for added modularity of the filtering. With the first implementation Toledo's bandpass filters were not used and instead the previous three samples were averaged as a method of filtering to remove sample to sample noise.

\section{Zero Crossing Demodulator}
Building on the strict zero crossing this zero crossing demodulator tried to use some more intelligence in finding the zero crossings through additional processing. One reason that the strict zero crossing approach was thought to have relatively poor results was due to the previously introduced challenge of DC offset. If the signal doesn't actually cross zero then it will be very hard to find the zero crossings. This method keeps a window of history, it was arbitrarily chosen to be one bit period, and from this collection of samples the average is taken to use this as the baseline - or zero value. Instead of checking to see if the signal crosses zero, the signal is analyzed for going from either above to below or below to above this average value. This ended up having worse results than the strict zero crossing demodulator. This was due in part to the fact that 2200Hz signals even when properly centered around zero will not have an average of zero since it does not complete two fill periods within on bit period, tainting the average.

\section{Windowed Zero Crossing Demodulator}
With now having a good handle on utilizing zero crossing a new approach was taken to keeping history. Instead of using the history to calculate where "zero" is, what if how many zero crossings are within one period are observed. If a windows slightly shorter than one bit period is selected, then if there are only two crossings within that window it will correspond to a 1200Hz symbol being present. More crossings than two means that a 2200Hz symbols must be present. The thought behind taking this approach is that it would give some additional resiliency to noise by finding the average during that bit period through utilizing multiple zero crossing instead of individually analyzing every zero crossing. 

\section{Peak Detection Demodulator}
After making a simple zero crossing overly complicated, it was decided that maybe a different approach should be taken, specifically to look at a different part of the signal. It was considered that perhaps better performance could be achieved by looking at the peaks in the signal instead of the noisy zero crossing around ground, or not around ground if there are DC offset problems. The nice thing about this is that the difference between two consecutive peaks will be equal to the period of the underlying signal. Although the methodology is the same as the zero crossing for converting the period to the actual frequency it was perceived that this would give better results. It turns out that this method did not work as well as hoped due to the fact that local peaks were commonly discovered from the noise instead of the actual peak in the transmitted signal.

\section{Derivative Zero Crossing Demodulator}
After a failure with the peak detection demodulator a new approach was taken to finding "peaks." Instead of actually looking for the peaks, the zero crossing demodulator was revisited with a new spin. Instead of using the raw samples for determining the frequency using zero crossings, the derivative was to be used. The derivative was calculated by doing the same averaging as in the strict zero crossing approach and then subtracting the current average from the average two samples ago. It was thought that this would solve the DC offset problem for sure, but it turns out that this was not the larger problem. The problem was with using the zero crossing approach and this derivative implementation ended up just having very similar results to the strict zero crossing.

\section{Goertzel Filter Demodulator}
Finally moving away from approaches utilizing zero crossing methodologies, an approach using a Goertzel filters was implemented. The implementation was very simple and corresponds with that outlined in the Demodulation Techniques Chapter. Since it has to be applied onto a set of data, originally a window size was selected that was equal to one bit period so as to make sure that the data being processed was only that of one frequency, but after analyzing the effect of the window size on performance, a window size of slightly bigger than a bit period ended up being better. The optimal size was tested to be 135 percent of a bit period, and the reason why this worked better is because it gave more signal in the window for the filter to lock onto and essentially the window was only extended 18 percent on each side of a bit period. This over extension of the window is what led to being able to exceed the performance of the original correlator on unfiltered data.

\section{Phase Locked Loop Demodulator}
Next, the PLL demodulator was implemented. Using Lutus's python based software PLL initial testing was performed to see how it would work for tracking AX.25 signals \cite{Lutus2011}. Once the parameters were tuned sufficiently that it seemed to be staying locked onto the signal it was ported over to java and actually run as a demodulator. Once inside of the javAX25 framework additional tuning was done programatically instead of manually to further fine tune the performance. The final results were that it was not the winner, but comperable to the other top contenders, correlation and Goertzel filter.

\section{Mixed Preclocking Demodulator}
Finally with numerous simple algorithms implemented, or at least they may appear that way due to their relatively few lines of code, it was time to try something much more complicated. Something that would only be possible in software to see if it would shine. This approach and name preclocking comes from an abbreviation for predetermined clocking where packets are analyzed a whole packet at a time. The start and end are found and then the clocking and hence bit boundaries are predetermined before the actual demodulation takes place on a bit by bit basis as opposed to a sample by sample basis. Each one of the preceeding algorithms was on a sample by sample basis, meaning they had to make their best determination of bits elapsed using a little bit of history.

There are five different steps to the demodulation in the Mixed Preclocking Demodulator. First, flags are found in the signal so that the demodulation can happen one packet at a time instead of just blindly trudging forward through the packet sample by sample. Second, the derivative of the whole packet is taken to  determine the zero crossings. Third, frequency transitions are extrapolated from the derivative data. Fourth, the frequency transitions found in the packet are used to determine the clocking or bit boundaries. Finally, fifth the tone demodulation is done on a baud by baud basis. It was speculated that processing one packet at a time with the correct clocking to demodulate bit by bit would allow for very accurate demodulation.

Although, it was hoped that the results would be better, there were so many different methodologies being used that it was very difficult to tune. For instance the flags were found using the correlation approach, and the transitions using a derivative, and the final demodulation using the zero crossings. What were thought to be the advantages ended up being the challenges, but as predicted it did pretty well to still be considered one of the successful implementations. The intricate nature of this demodulator made it delicate which was noticed during the testing through the fact that it would not decode any packets unless a bandpass filter was used on the incoming data.

\section{Goertzel Preclocking Demodulator}
After the first attempt as a preclocking approach, it was thought that perhaps only using one methodology to perform all the different steps of demodulation would be at the very least simpler, and hopefully better. The perception that it might be better came from the fact that there was only one item to tune, the Goertzel Filter. Instead of having to worry about noise affecting zero crossings and the derivative potentially adding emphasis problems, only the filter has to be considered. Unfortunately the number of packets that this method decoded was not as many as the first Goertzel approach, or the previous preclocking. This was due to the fact that even though there was one underlying algorithm it was used in three separate instances, and each wanted slightly different tuning. The three instances were for flag detection, frequency transition detection, and the the final bit by bit demodulation.

\section{Goertzel Exhaustive Precklocking Demodulator} %TODO fill in numbers below...
The final algorithm implemented was just a manner of verification, and another one that could only be performed in software. Instead of analyzing packets one at a time using flags as the start and end points, a whole array of data that had a length equal to the number of samples that a packet if the maximum length would have. Every time a few more samples came in, every single clocking was attempted on the large array of data just to see what packets could be decoded by exhaustively searching for data. The performance of this algorithm in terms of time to run was much longer. For instance the mixed preclocking and original Goertzel preclocking took XXX and XXX to run respectively while this exhaustive search took XXX on the 25 minute 49 second Track 1 of the test suite. This means that although a 2.1Ghz Intel i7 (i7-3612QM) could keep up in real time it is close enough that any kind of embedded system at the time of this writing would not have the computational power to keep up with live data. Gratefully, this approach only decoded an additional 15 packets that the Correlation, original Goertzel (non-preclocking), and PLL did not decode. This result could be used to make the argument that the few more packets produced is not worth the vast number more CPU cylces it take to acheive it.




\end{document}
