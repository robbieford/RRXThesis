\chapter{Origins of ax.25}
%Currently copied from old background chapter
APRS is a local area awareness network designed from hams (amateur radio operators), by hams. There is a plethora of features supported within the specification of APRS including the sending of messages, bulletins, meteorological data, waypoints, objects, and most commonly GPS locations. Many of the capabilities have not been fully implemented and other implementation details are inconsistent. The goal of this research is not to discuss the shortcoming in the APRS protocol itself which lies within its implementation and use on top of the AX.25 protocol. Instead this research focuses on Layer 1 problems in the Open System Interconnection (OSI) model.

The Layer 1 protocol for ARPS is based off of a Bell 202 modem. The Bell 202 modem was patented in 1984 using 1200Hz and 2200Hz tones, although the patent was originally filed in 1981 [1]. Interestingly, the international telecommunication union didn�t publish a standard for these modems that were used in telephone networks until 1988. In the standard, however they use 1300Hz and 2100Hz tones for symbol 1 (called a mark) and symbol 0 (called a space) respectively [2]. The basic modulation scheme is that the marks and spaces are used in a non-return to zero inverted (NRZI) encoding for the actual data. When a transition occurs from one symbol to another that symbolizes a �0� bit in the original data bit stream and if the symbol remains constant over multiple symbol periods, that signifies a �1� bit in the original data bit stream. It is also worth noting that there will be no more than 5 consecutive symbols during a packet, as the scheme will bit stuff in order to make sure that synchronization on the clock can be maintained. The only exception to this rule is for the flag(s) that mark the start and end of a packet. The AX.25 flag is hex 0x7E without bit stuffing. Common practice is to send multiple flags consecutively to give transmitting radio time to key up and settle and to give receiving radios time for their squelch to open.

This encoding and modulation is used to transmit the data of the APRS packets, but further background is needed in the APRS environment to discuss where this research applies. As mentioned this will not be looking into the APRS protocol itself but more focused on the Bell 202 modem and specifically the demodulation aspects. Implementing a Bell 202 modulator is simple in comparison to making a robust demodulator - especially when trying to implement one in software.

\section{High-Level Data Link Control}

\section{Frequency Shift Keying}

%Perhaps a little tangental and out of the scope?
\section{Radio Teletype}

\section{Bell 202}

%Apparently similar to x.75?
\section{x.25}

\section{ax.25}
