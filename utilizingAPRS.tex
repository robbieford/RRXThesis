\chapter{Approaches to Utilizing the APRS Network}
In the world of APRS there are many solutions that hams take advantage in order to utilize the network. Some they find and make work, some they purchase to use exclusive for APRS, and some go through the trouble of inventing their own solutions. This chapter explains some of the common systems used on the APRS network, primarily those that can be used for receiving, starting with terminal node controllers and progressing to software based demodulation.

\section{Terminal Node Controllers}
Currently there are many systems that will demodulate these Bell 202 encoded APRS packets. The original hardware used for this communication style was dedicated modems similar to dial-up 56k modems that did the encoding and decoding. These modems were connected directly to the radio and the radio would let the modem know through a signal pin if the radio was receiving what it thought was a data carrying signal. And conversely, the terminal node controller (TNC), a modem used in AX.25 operation, would let the radio know by signalling the radio to transmit when it had data to send out \cite{Wolfgang2005}. Just as a reminder of the age of the technologies that are being used for the data transport of APRS, packet radio originated around 1985 as TNCs became affordable \cite{Helms1992}. This means that this technology is 30 years old at the time of writing in the year 2015.

With a radio and a TNC, amateur packet stations and digipeaters (digital packet repeaters) are possible \cite{Group2012,Wiki2012}. Digipeaters are an essential part of the ham packet network, but many users wish to report their GPS position onto the APRS network instead of just relaying traffic for other stations. In order to accomplish this, a GPS receiver is required. Now, stations can take the data from their GPS receiver and put it in the payload of the APRS packet and transmit the GPS reported position onto the network. One other common use of a TNC is an internet attached digipeater, usually though the use of a computer, that would allow the data to be posted to the APRS-IS servers \cite{Community2015}. Just to give an example of some of the TNCs available, those within the testing testing scope of this project will be listed. There are eight - six unique models - whose decoding results are compared to the software approaches. These modems included the two AEA PK-88s, two Kantronics KAM Pluses, a Kantronics KAM, an MFJ-1278, a PK-232, and a PK-232MBX. An example image of what a TNC looks like, the photo features a KAM Plus can be seen in Figure \ref{kantronicsKamPlus}. Although these modems work for decoding and encoding Bell 202 packets, this is not their only purpose as they are multiple mode modems and support numerous other transmission modes and modulation schemes, so just a note that these are modems that happen to be used for and work with APRS.

\begin{figure}
  \centering
	\includegraphics[width=0.75\linewidth]{images/Kantronics-KAM-Plus.jpg} 
	\caption{Image of the Kantronics Kam Plus \cite{KK6RF}}
   \label{kantronicsKamPlus}
\end{figure}

\section{Specialized APRS Hardware}
Many people know exactly what they would like to do with APRS and exactly what traffic they want to contribute to the APRS network. This has allowed for companies to start commercially making dedicated APRS hardware, since there is a demand for it. In addition to making this hardware available the producers support the hardware and make pretty user interfaces for the users to be able to program the hardware exactly as they like and without having to invest much time into understanding how different components work together. Some examples of APRS exclusive devices are ArgentData�s OpenTrackers (Figure \ref{openTracker3}), Byonics� TinyTrack, and Fox Delta�s Fox Track \cite{Miller,Byonics,Foxtrak}. These compact packages along with a radio and a GPS module perform APRS tasks at a satisfactory level for many users.

\begin{figure}
  \centering
	\includegraphics[width=0.75\linewidth]{images/Ot3m-termblk.jpg} 
	\caption{Image of Argent Data's Open Tracker 3. \cite{Data}}
   \label{openTracker3}
\end{figure}

Since the average user only wants to report positional information, these dedicated devices are simple to setup to do such but also only include a simple feature set. Although these trackers contain some features, since they are all small embedded systems they can not and do not have implemented all of the features that APRS supports. An example is the messaging service. Since these devices don�t have a display or a keypad, there is no way to input or display a message. Certain radio manufacturers have begun to integrating the TNCs into the radios themselves to utilize the radio�s screen. The Kenwood TM-D700 series and Yaesu FTM-350 (Figure \ref{yaesuFTM350}) are examples \cite{Kenwood,Yaesu}.

\begin{figure}
  \centering
	\includegraphics[width=0.75\linewidth]{images/FTM-350US_F.jpg} 
	\caption{Image of the FTM-350 Radio which has APRS integrated. \cite{Yaesua}}
   \label{yaesuFTM350}
\end{figure}

However, both the options in this section and the one previous on TNCs require going out and buying special hardware in order to perform APRS whether it be a TNC itself of a OpenTracker. This can be expensive and cost prohibitive for some hams to be able to begin APRS operations.

\section{Software Based Demodulation}
It can be assumed that before a ham operator becomes interested in the APRS network and sending APRS packets that they will already have a radio. So, if they already have a radio all they have to do is buy a piece of hardware that will do the modulation in order to send a packet. However, hardware costs money and before diving right in it might be nice to get their feet wet first. A good, cheap alternative to dedicated hardware is to use hardware that hams already have. A good choice that will fit the needs is a computer, which most hams probably own at least one of. On this computer amateurs can build or buy a cheap interface to a radio, around \textasciitilde\$15 instead of \textasciitilde\$150 for a piece of dedicated hardware, and then use software to do the modulation and demodulation.

This seems to be a route that some are taking and a demodulation scheme that this project explores in detail, but first some more information on current systems that operate in this software realm. Some examples of the software that can be used are George Rossopoylos�s Packet Engine \cite{Rossopoylos}, Thomas Sailer�s Linux Sound Modem \cite{Sailer1997,Sailer2000}, or Sivan Toledo's javaAX25 \cite{javax25github, Toledo2012a}. On a computer, even ones with minimal resources, there are algorithms that are being used to demodulate the APRS packets. Again, what this project aims to investigate is what improvements can be made to the algorithms and software based demodulation approaches in order to decode these packets in a more robust fashion and to try and get similar performance to TNCs and dedicated hardware. This is based on observations in initial analyses where software was unable to decode packets, the hypothesis was made that improvements can be made to software based demodulation. It is worth mentioning here that the Toledo's javaAX25 will be the software basis that will be developed on top of, so more information is to follow.

\subsection{javaAX25}
Sivan Toledo's javaAX25 is one method of utilizing the APRS network. Toledo's software is very comprehensive in handling the encoding, decoding, radio control, and interfacing with sound cards to allow for full use of APRS using this software. However in addition to just being able to utilize APRS there is also a test application inside of this package that allows for quick and easy testing of everything in the suite - of the most interest, however is the ability to be able to test demodulators. Although all of these features were included, the three primarily used in this project were the modulation, demodulation, and demodulator testing. Due to its extent and its ease of access on line through Github, this was chosen to be the basis for this project \cite{javax25github}. For a complete list of features the manual can be found in the following reference, and even from the beginning the mission statement that he outlines coincides with that of this research \cite{Toledo2012a}.

Toledo did some benchmarking of his software and found that running two demodulators in parallel provided the best results. The demodulators were exactly the same, the only difference is that one was processing data after a bandpass filter that was just centered around the two frequencies of interest, and the other had a bandpass filter that in additionally applied 6dB of attenuation at 1200Hz \cite{Toledo2012}. Being published in 2012 this is the newest reference in this paper on the subject of AX.25, which provided additional incentive to use this project for this research. As added verification of making the correct decision of what software to use, a very popular Android APRS application written by Georg Lukas uses javaAX25 by a direct import \cite{APRSdroid}.