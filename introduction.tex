\chapter{Introduction}

Amateur Radio Operators, commonly refereed to as hams make the best of what resources they have available to them. However, once something is working a "don't touch it if it ain't broke" approach is often taken. Between these two mentalities some interesting phenomenon have occurred within the ham community. For example, some radio systems that are in active use today have only seen very minimal attention since the 19080s when they were originally installed. On the flip side of leaving things alone, hams are very quick to take advantage of a good deal or opportunity when they are presented one.

A recent scenario that expresses both of these characteristics of hams is when the Federal Communications Commission (FCC) required that public service agencies such as police, fire, and ambulance go to narrow banding (a 12.5kHz channel from a 25kHz channel). This change was needed to be able to support more channels in the limited radio frequency spectrum \cite{Commission2012}. Due to this change a lot of equipment became available that could not do narrow banding, which many hams sprung at the opportunity for. In addition to showing their need to take advantage of a good opportunity it also shows that they do not feel the need to upgrade their systems even though there are congested areas that would benefit immensely from narrow banding.

The implementation and development of the Automated Packet Reporting System (APRS) is no exception to the way hams approach things \cite{Bruninga}. Much of this system is based off old hardware that was readily available and few improvements have been made. Although it is nice to have a stable specification as technology evolves it would be nice if new features were added. So, what is APRS, and why does it matter? A brief introduction to APRS is that is is a digital communication scheme used by hams where a packet (whose content is varied, but is usually a GPS position - which gives APRS its nickname the Automated Position Reporting System) is sent out over radio. A major challenge to this protocol and method of digital communication is the fact that it uses radio, which is susceptible to interference and weak signals as the distance from the transmitting station increases. This research focuses specifically on the receiving end of these signals in order to see what improvements can be made to software based approaches to decoding - demodulating - these packets.

The reasoning for trying to make improvements in software based demodulation are many, but a few of the more motivational ones are to follow. One advantage of doing software based demodulation is it removes the necessity of specialty hardware. Instead of having dedicated hardware whose sole purpose is to modulate and demodulate APRS packets, hams can use a computer to do these tasks. Using a computer's sound card, audio from the radio can be processed using software to decode received packets, or audio can be played from the sound card to the radio to be transmitted. With the abundance of personal computers this can provide a much cheaper solution for hams who are interested in getting their feet wet trying out APRS without having the scare of putting down a potentially big initial investment (~\$200 \cite{Kantronics2014,Outlet2014}) for a piece of hardware that serves one purpose.

Another cost advantage is when multiple APRS networks are operated alongside each other. For some events that hams assist with, GPS tracking is used to keep track of assets; support vehicles, ambulances, water trucks, etc. In order to get more frequent position updates the traffic can be moved from the primary transmit frequency to a different back haul frequency to alleviate some of the RF congestion. If there are multiple back haul frequencies, say three of them, in addition to the primary transmitting frequency there are a total of four frequencies carrying APRS traffic that need to be monitored and then the traffic on those frequencies demodulated. Instead of spending \$800 to get four dedicated units for APRS to handle all of the traffic an extra sound could be purchased for a computer (~\$20 \cite{Newegg}). Since audio inputs are typically stereo with a left and a right channel, these can be considered 2 separate audio inputs since the radio transmissions are mono (single channel). Hence, between the input that the computer already has and the new one added through the extra sound card there could be a total of 4 audio input channels on the computer which the audio from the radios can be piped into.

In addition the the price advantages of software based demodulation approaches there is also one other primary advantage. If software is being used instead of hardware there is the potential for a lot more capabilities since processing power and available memory go up drastically. For instance on one of the dedicated hardware solutions, the Kantronics KPC-3 Plus, it has a whopping 512KB of memory compared to that of any computer which is over 4GB as of 2014 - and that is just the ram, not the hard drive space \cite{Kantronics2014,Graham-Smith2014}. Additionally instead of just being able to handle live events and process each data point as well as possible as soon as it comes in, post processing becomes an option.

With the cost and versatility of a software demodulation solution now introduced the paper will flow as follows: Chapter 2 goes into background information, with a deeper introduction to APRS and a presentation of the aspects important to understanding this research. Some of the current methods for interfacing with APRS, both hardware and software, are explained in Chapter 3. Demodulation techniques are discussed in Chapter 4. Chapter 5 talks about the challenges of demodulating APRS packets. Chapter 6 discusses the methods used for benchmarking and comparing the demodulators. In Chapter 7 information on how the demodulators and algorithms are tested is presented. Chapter 8 goes into more detail about the implementations in this project. Chapter 9 discusses the results of both the newly implemented algorithms and compares them to other demodulators. Areas of additional research and future work are discussed in Chapter 10. Chapter 11 is concluding remarks.
