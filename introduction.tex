\chapter{Introduction}

Amateur Radio Operators, commonly referred to as "hams," make the best of what resources they have available to them. However, once something is working a "don't touch it if it ain't broke" approach is often taken. Between these two mentalities some interesting phenomenon have occurred within the ham community. For example, some radio systems that are in active use today have only seen very minimal attention since the 1980's when they were originally installed. On the flip side of leaving things alone, hams are very quick to take advantage of a good deal or opportunity when they are presented one.The implementation and development of the Automated Packet Reporting System (APRS) is no exception to the way hams approach things\,\cite{Bruninga}. Much of this system is based off older hardware and protocols - from the 1980s - that was readily available and few improvements have been made, and although the specification has been relatively stable there are inconsistencies. These inconsistencies include varying implementations from vendor to vendor as well as portions of the specification that are not clearly defined resulting in vastly inconsistent performance\,\cite{KWFThesis, KWFTAPR}.

%A recent scenario that expresses both of these characteristics of hams is when the Federal Communications Commission (FCC) required that public service agencies such as police, fire, and ambulance go to narrow banding (a 12.5kHz channel from a 25kHz channel). This change was needed to be able to support more channels in the limited radio frequency spectrum\,\cite{Commission2012}. Due to this change a lot of equipment became available that could not do narrow banding, which many hams sprung at the opportunity for. In addition to showing their need to take advantage of a good opportunity, it also shows that they do not feel the need to upgrade their systems even though there are congested areas that would benefit immensely from narrow banding.

So, what is APRS, and why does it matter? A brief introduction to APRS is that it is a digital communication scheme used by hams where a packet (whose content is varied, but is usually a GPS position - which is what gave APRS it's original name "Automated Position Reporting System"\cite{WikiAPRS}) is sent out over radio. A major challenge to this protocol and method of digital communication is the fact that it uses radio, which is susceptible to interference, weak signals as the distance from the transmitting station increases, as well as a myriad of other items. This research focuses specifically on the receiving end of these signals in order to see what improvements can be made to software based approaches to decoding - demodulating - these packets.

The reasoning for trying to make improvements in software based demodulation are many, but a few of the more motivational ones are to follow. One advantage of doing software based demodulation is that it removes the necessity of specialty hardware; Instead of having dedicated hardware whose sole purpose is to modulate and demodulate APRS packets, hams can use a computer to do these tasks. By using a computer's sound card, audio from the radio can be processed using software to decode received packets, or audio can be played from the sound card to the radio to be transmitted. With the abundance of personal computers, this can provide a much cheaper solution for hams who are interested in trying out APRS without having to put down a potentially big initial investment (\textasciitilde\$200\,\cite{Kantronics2014,Outlet2014}) for a piece of hardware that serves one purpose. The price of this specialty hardware is steep and it is limited to only performing communication on a single channel. When using a line in / out on a computer they are typically stereo meaning that a single sound card could handle operations on multiple channels. If two channels just is not enough the capabilities of a computer demodulator can be expanded my merely adding another sound card which is relatively cheap at \textasciitilde\$20\,\cite{Newegg}. To perform communication on 4 channels using dedicated hardware the cost would be $800! For this cost a whole computer with a half dozen sound cards could be purchased, only further expanding capabilities.

%Another cost advantage is when multiple APRS networks are operated alongside each other. For some events that hams assist with, GPS tracking is used to keep track of assets, such as support vehicles, ambulances, water trucks, etc. In order to get more frequent position updates the traffic can be moved from the primary transmit frequency to a different back haul frequency to alleviate some of the Radio Frequency (RF) congestion. If there are multiple back haul frequencies - three, for example - in addition to the primary transmitting frequency, there would be a total of four frequencies carrying APRS traffic that need to be monitored and the traffic demodulated. Instead of spending \$800 to get four dedicated units for APRS to handle all of the traffic an extra sound card could be purchased for a computer (\textasciitilde\$20\,\cite{Newegg}). Since audio inputs are typically stereo with a left and a right channel, these can be considered as two separate audio inputs since the radio transmissions are mono (single channel). Hence, between the input that the computer already has and the new one added through the extra sound card, there could be a total of four audio input channels on the computer which the audio from the radios can be piped into.

In addition the the price advantages of software based demodulation approaches there is also one other primary advantage. If software is being used instead of hardware there is the potential for a lot more capabilities since processing power and available memory increase drastically. For instance, on one of the dedicated hardware solutions, the Kantronics KPC-3 Plus, has a mere 512KB of memory compared to that of any computer which is over 4GB as of 2014 - and that is just the ram, not the hard drive space\,\cite{Kantronics2014,Graham-Smith2014}. Additionally, instead of just being able to handle live events and process each data point in the best manner possible as soon as it comes in, post processing becomes an option.

With the cost and versatility of a software demodulation solution now introduced, the paper addresses the following: Chapter 2 goes into background information, with a deeper introduction to APRS and a presentation of the aspects important to understanding this research. In Chapter 3, some of the current methods for interfacing with APRS, both hardware and software, are explained. Demodulation techniques are discussed in Chapter 4. Chapter 5 talks about the challenges of demodulating APRS packets. Chapter 6 discusses the methods used for benchmarking and comparing the demodulators. In Chapter 7, information on how the demodulators and algorithms are tested is presented. Chapter 8 goes into more detail about the implementations in this project. Chapter 9 discusses the results of both the newly implemented algorithms and compares them to other demodulators. Areas of additional research and future work are discussed in Chapter 10. Chapter 11 is concluding remarks.
