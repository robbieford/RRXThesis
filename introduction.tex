\chapter{Introduction}

%New introduction?
Amateur Radio Operators, commonly refereed to as hams make the best of what resources they have available to them. However, once something is working a "don't touch it if it ain't broke" approach is often taken. Between these two mentalities some interesting phenomenon have occurred within the ham community. For example, some radio systems that are in active use today have only seen very minimal attention since the 19080s when they were originally installed. On the flip side of leaving things alone, hams are very quick to take advantage of a good deal or opportunity when they are presented one.

A recent scenario that expresses both of these characteristics of hams is when the Federal Communications Commission (FCC) required that public service agencies such as police, fire, and ambulance go to narrow banding (a 12.5kHz channel from a 25kHz channel). This change was needed to be able to support more channels in the limited radio frequency spectrum \cite{Commission2012}[INSERT REFERENCE HERE... http://transition.fcc.gov/pshs/public-safety-spectrum/narrowbanding.html ]. Due to this change a lot of equipment became available that could not do narrow banding, which many hams sprung at the opportunity for. In addition to showing their need to take advantage of a good opportunity it also shows that they do not feel the need to upgrade their systems even though there are congested areas that would benefit immensely from narrow banding.

The implementation and development of the Automated Packet Reporting System (APRS) is no exception to the way hams approach things. Much of this system is based off old hardware that was readily available and few improvements have been made. So, what is APRS, and why does it matter? A brief introduction to APRS is that is is a digital communication scheme used by hams where a packet (whose content is varied, but is usually a GPS position - which gives APRS its other nickname Automated Position Reporting System) is sent out over radio. A major challenge to this method of digital communication is the fact that it uses radio, which is susceptible to interference and weak signals as the distance from the transmitting station increases. This research focuses specifically on the receiving end of these signals in order to see what improvements can be made to software based approaches to decoding - demodulating - these packets. 

The paper flow will be as follows: Section 2 goes into background information, with a deeper introduction to APRS and a presentation of the aspects important to understanding this research. Some of the current demodulation approaches, both hardware and software, are explained in section 3, as well as a deeper explanation of the motivation behind this project. Section 4 talks about the challenges of demodulating APRS packets. Section 5 discusses the methods used for benchmarking and comparing the demodulators. Sivan Toledo's javax25 is introduced in section 6. The new implementation details are presented in section 7. Section 8 goes into more detail of how the software algorithms were tested. Section 9 discusses the results of the newly implemented algorithms and compares them to other demodulators. Areas of additional research and future work are discussed in section 10. Section 11 is concluding remarks.
