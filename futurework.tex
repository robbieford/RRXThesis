\chapter{Future Work}
The sections in this chapter will explain some aspects that this project uncovered that have the potential for future research. There are three main items that the author thinks should be explored in more depth as future work in this area of AX.25 software based demodulation, including the Discrete Short-Time Fourier Transform, the use of the checksum for forward error correction, and to actually integrate these demodulators with live traffic from a radio.

\section{The Discrete Short-Time Fourier Transform}
In a paper by Zhonghui-Chen et. al., methods are outlined to use the discrete short-time Fourier transform (DSTFT) to demodulate a binary frequency shift keyed (BFSK) signal. After a discussion of the DSTFT they go into their specific implementation, but this appears to show promise because of there results section. Granted, this was for a simulation but they showed that the error rate was lower using the DSTFT than traditional coherent demodulation even for lower signal to noise ratios \cite{Chen2008}

\section{Use the CRC for FEC}
Each AX.25 packet contains a checksum that is generated by using a cyclic redundancy check (CRC). This is used by all demodulators in order to determine if the packet that the demodulator thinks it decoded was actually a legitimate packet or just noise that happened to look like a packet. Although the CRC was not intended for forward error correction (FEC), it would be interesting to see the effects of using it as such. With algorithms such as the preclocking algorithms the power of each symbol is determined and this power could be used to assign a level of confidence on that demodulated bit. If a packet fails the CRC check, and all except for one of the bits has a confidence greater than 80 percent, would the CRC pass if that one bit was flipped to the other symbol? This is a very good argument for the use of software since this meta data about the decoding of the packets can be kept in memory, something that most hardware is very light on. 

\section{Integrations with a Radio}
Finally another area for future work would be to integrate the new algorithms with a radio and use them to decode live data. Although the data used in the testing was a recording of live data, it would be very gratifying to be able to see these algorithms decode audio straight from the radio. Inside of javAX25 the packages should already support it, so it should be a matter of just setting it up and letting it run. This analysis would allow for verification of the implementations functionalities as well as to be able to see how the different algorithms perform side by side in real time.

%Edge detection?
